The overall number of electric vehicles (EVs) on the roads is increasing and the need for the safety and reliability of their batteries is becoming more important. Dendrite formation is one of the most critical issues facing an electric vehicle's battery and can cause not only safety hazards but also destruction of the battery itself. This project offers a unique procedure for spotting battery dendrite risk early, using a hybrid machine learning model that merges both classical and quantum computing methods. Quantum Support Vector Machines classify the health of the battery time-series data, which consists of voltage, current and temperature. The sensors that are placed on the battery gather continuous data for a certain period of time and this data is then divided into smaller segments using a sliding window technique of pre-processing. For each segment, various statistics including average, variance and rate of change are computed to identify potential features in the battery's behavior. Features that have been selected are subsequently provided to the hybrid model where the QSVM further pushes the feature transformation process making it easier to detect very slight abnormalities which are signs of dendrite growth. It is the objective of this model to generate real-time, very precise predictions of dendrite risk that will facilitate quick actions taken and thus better management of the overall battery. The fusion of quantum computing and traditional machine learning techniques is what marks a key milestone for the EVs battery health monitoring sector. 
