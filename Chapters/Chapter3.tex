% Chapter 3

\chapter{DESIGN} % All Chapter Headings in ALL CAPS

\section{Design Overview}
The proposed system design aims at coming up with a workable and dependable hybrid Quantum Support Vector Machine (QSVM)-based system in detecting anomalies and assessing risks in electric vehicle (EV) battery charging systems. The architecture is inspired by currently available quantum machine learning-based battery anomaly detection solutions, especially those based on an unsupervised clustering algorithm like K-means. Although the clustering-based methods have proven to be useful in finding rough deviations, they fail to perceive fine or earlier deviations and lack a decisional margin. The proposed system, to overcome such shortcomings, substitutes the clustering process with an explicit risk-oriented decision layer and the margin-based hybrid QSVM classifier.

The system is designed based on modular pipeline architecture where every module has a particular role to play and the structured outputs are sent out to the next process. The approach is appropriate in the real world because this modular design enhances interpretability, scalability and easy integration with the currently available battery management systems.

\begin{landscape}
\begin{figure}[p]
    \centering
    \includegraphics[width=\linewidth]{Figures/diagram1.pdf}
    \caption{Architecture of the Proposed QSVM-Based EV Battery System}
    \label{fig:Architecture of the Proposed QSVM-Based EV Battery System}
\end{figure}
\end{landscape}
\section{System Overview}

The time-stamped EV battery sensor data is processed in the following modules according to the proposed system: data acquisition, preprocessing, classical feature engineering, quantum-inspired feature mapping, hybrid QSVM classification, risk assessment, and output monitoring. Data preparation and baseline feature extraction is done using classical machine learning methods, whilst the representation of features is improved using quantum-inspired learning principles. A QSVM-based classifier is used to produce the final decision in a form of a risk-level in battery charging.

The proposed design uses a supervised or semi-supervised QSVM as opposed to the base reference work which has been using the unsupervised clustering method to detect anomaly. It allows clearly defined decision boundaries, better noise sensitivity, and more dependable detection of early warning features associated with unsafe charging behavior.

\subsection{Design: Data Acquisition Module}

The data acquisition module is the interface of the system and gathers time stamps of sensor data of EV battery packs through charging cycles. A set of synchronized readings of voltage, current, temperature, and state of charge (SOC) is stored in each record of the data.

This is a multivariate time series representation of battery behavior. Mitigation Multiple parameters interact with each other to form many of the battery anomalies and not single threshold violations. This module is as per the standard battery monitoring practices and comparable to the data collection methodology as employed by the base framework.

\subsection{Design: Data Preprocessing Module}

Raw sensor data can be noisy, have outliers, and dissimilar scales between parameters. In order to solve these problems, the preprocessing module is used to filter noise, normalize and segment the time.

Noise filtering cushions the effects of sudden upheavals and errors by the sensors. Normalization brings all sensor parameters in similar range such that, no single feature takes control of the learning process. Then a sliding window segmentation method is used to subdivide the continuous time-series data into fixed length windows. Each of the windows reflects short term charging dynamics and it is an independent learning sample.

This preprocessing methodology allows analysis of time behavior and adheres to common time-series processing procedures in battery monitoring systems.

\subsection{Design: Classical Feature Engineering Module}

After preprocessing, classical feature engineering is performed to obtain meaningful descriptors of every time window. Statistical characteristics like mean and variance are used to represent the general operating characteristics and temporal characteristics like change of rate and stability measure the dynamism of charging behavior.

A set of these features is represented as a classical structured feature vector of each window. The module is able to give an interpretable and computationally efficient baseline representation and is equivalent to the classical feature extraction step in the base paper. It also removes the dimensionality of unprocessed sensor measurements, which is convenient to process later with quantum-inspired algorithms.

\subsection{Design: Quantum-Inspired Feature Mapping Module}

The primary contribution of the given system is the quantum-inspired feature mapping module. This module converts the classical features instead of clustering or classifying them directly, into a higher-dimensional Hilbert space, via quantum-inspired encoding methods.

At the initial step, the superposition-based representation is used to encode classical feature vectors, where several components of features can be expressed at the same time. The second stage is an entanglement-inspired transformations that are used to model feature interactions. Such interactions are essential in EV battery systems, in which interacting temperature, voltage, and current are often the cause of unsafe charging behavior.

Even though many of these transformations are runnable on classical hardware, they are based on the mathematical form of quantum state representation, leading to the richer and more expressive feature space without needing physical quantum computation.

\subsection{Design: Hybrid QSVM Classification Module}

The proposed system uses a hybrid QSVM classifier to detect the existence of anomalous charging behavior. In contrast to the base paper that employed K-means clustering algorithm to learn the decision boundary, the QSVM learns the margin-based decision boundary, which divides normal and abnormal charging patterns in the quantum-enhanced feature space.

The QSVM is a quantum-inspired kernel with the classical SVM optimization. This hybrid architecture maintains the stability and scalability of traditional SVMs and enjoys the improved feature representation. The QSVM has a higher ability to resist noise, fewer rare anomalies, and more distinct separation between normal and abnormal behavior than any of the clustering-based algorithms.

\subsection{Design: Risk Assessment and Outputs}

The system does not simply label anomalies, instead having a risk assessment module. The QSVM margin or anomaly score is assessed with the help of predetermined thresholds in order to determine each charging window as a High Risk or Low Risk one. Such a distinction between the detection and interpretation is more informative and conforms to the real-world battery management needs.

The risk level and a confidence score are the ultimate output and allow anticipating the potential danger and preventive measures in the case of unsafe charging conditions.

\subsection{Design Summary}

Overall, the proposed construction can be seen as an extension of current quantum machine learning-based battery anomaly detection models by introducing quantum-inspired feature mapping in a classical learning framework and substituting the decision logic based on clustering with a hybrid QSVM classifier. The modular architecture supports a high level of detection of anomalies at the same time being interpretable, computationally efficient, and feasible in practice to the applications of EV battery safety.

