% Chapter 4

\chapter{IMPLEMENTATION} % All Chapter headings in ALL CAPS

\section{Accuracy}
\paragraph{\textbf{Accuracy}}
measures the proportion of correctly classified samples among all test instances.
\begin{equation}
\mathrm{Accuracy} = \frac{TP + TN}{TP + TN + FP + FN}
\end{equation}
where TP, TN, FP, and FN denote true positives, true negatives, false positives, and false negatives, respectively. 

\section{Recall}
\textbf{Recall}
refers to evaluating the model’s skill to accurately classify real anomalous examples
\begin{equation}
\mathrm{Recall} = \frac{TP}{TP + FN}
\end{equation}

High recall is critical when it comes to safety monitoring for EV battery packs, as a lack of recognition of an anomaly can cause serious problems.

\section{Feature Space Seperability}
\textbf{Feature Space Seperability(FSS)}
This refers to the separability in the feature space that results after the quantum-inspired processing step
\begin{equation}
\mathrm{FSS} =
\frac{\| \mu_{\mathrm{normal}} - \mu_{\mathrm{anomaly}} \|_{2}}
{\sigma_{\mathrm{normal}} + \sigma_{\mathrm{anomaly}}}
\end{equation}

A larger FSS value means better class separability achieved by the quantum support feature transformation process.

\section{Variance Amplification Ratio}
\textbf{Variance Amplification Ratio(VAR)}
measures the amount of variance in features that is increased by the quantum-inspired transformation.
\begin{equation}
\mathrm{VAR} = \frac{\mathrm{Var}\!\left(Q(x)\right)}{\mathrm{Var}(x)}
\end{equation}
Q(x) means the quantum-transformed feature vector.

