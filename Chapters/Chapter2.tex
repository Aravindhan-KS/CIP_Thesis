\chapter{LITERATURE REVIEW}

\renewcommand{\arraystretch}{1.3}

% Center longtable properly
\setlength{\LTleft}{\fill}
\setlength{\LTright}{\fill}

\begin{longtable}{|c|p{0.22\textwidth}|p{0.26\textwidth}|p{0.20\textwidth}|p{0.24\textwidth}|}

\caption{Literature Review}
\label{tab:tab1} \\

\hline
\textbf{S.No} & \textbf{Paper} & \textbf{Methodology} & \textbf{Limitations} & \textbf{Ideas for Adoption} \\
\hline
\endfirsthead

\hline
\textbf{S.No} & \textbf{Paper} & \textbf{Methodology} & \textbf{Limitations} & \textbf{Ideas for Adoption} \\
\hline
\endhead

\hline
\endfoot

\hline
\endlastfoot

1 &
Comparative Investigation of Quantum and Classical Kernel Functions Applied in Support Vector Machine Algorithms &
Performs a comparative study of classical kernels (RBF, polynomial) and quantum kernels within the SVM framework. Different quantum feature maps are evaluated using QSVM on benchmark datasets. &
Uses quantum simulators; high kernel computation cost; limited scalability; lacks real-world application focus. &
Adopt the quantum kernel comparison framework to evaluate classical SVM versus QSVM for battery time-series features, justifying improved separability for early dendrite-risk patterns. \\

\hline

2 &
Assessment of Quantum Feature Map for Binary Forest Classification using QSVM &
Evaluates multiple Pauli-based quantum feature maps for binary classification. Features are normalized, reduced, encoded, and classified using QSVM with Pareto-front analysis. &
Relies on simulators; inconsistent QSVM superiority; scalability limitations in NISQ-era hardware. &
Apply quantum feature map selection strategies to battery sensor data and use Pareto analysis to balance accuracy and computational cost. \\

\hline

3 &
Hybrid Quantum–Classical Framework for Urban Traffic State Classification &
Implements a hybrid pipeline combining classical feature extraction with quantum encoding and QSVM-based classification. &
Limited dataset; simulator-based execution; real-time scalability not addressed. &
Use the hybrid classical–quantum architecture for battery monitoring by applying QSVM after classical preprocessing of voltage and temperature data. \\

\hline

4 &
AutoQML: A Framework for Automated Quantum Machine Learning &
Introduces automated quantum model selection, feature encoding, and hyperparameter tuning in hybrid workflows. &
High computational overhead; simulator dependence; not tailored for time-series data. &
Adopt automated quantum kernel and hyperparameter tuning to reduce manual QSVM optimization for battery datasets. \\

\hline

5 &
Real-Time Quantum Machine Learning-Based Anomaly Detection for Lithium-Ion Battery Packs &
Proposes quantum-enhanced K-Means clustering for real-time anomaly detection using battery sensor data. &
Focuses on clustering; requires multiple qubits; real-time hardware feasibility unclear. &
Extend the anomaly detection framework by replacing clustering with QSVM-based supervised dendrite-risk classification. \\

\hline

6 &
Quantum Kernel-Based Support Vector Machine with QAOA Embedding for Lung Cancer Prediction &
Uses QAOA-based quantum feature embedding with QSVM, achieving better results for small datasets. &
Simulator execution; high cost; evaluated only on static medical data. &
Adopt QAOA-based quantum embedding for capturing nonlinear patterns in reduced battery feature sets. \\

\hline

\end{longtable}