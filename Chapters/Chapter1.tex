% Chapter 1

\chapter{INTRODUCTION} % Write in your own chapter title All Chapter headings in CAPS
EVs can contribute to sustainable transportation and lithium-ion batteries are the key to their operation. The charging process subjects the EV batteries to different electrical and thermal conditions, which when allowed to get out of check may lead to abnormal behavior. Thermal runaway is one of the most important safety issues since the phenomenon, which is defined as an abrupt rise in temperature, may cause battery deterioration or system malfunction. Such abnormal conditions must therefore be identified early in order to be able to ensure that operations are safe and also to guarantee reliability of the system.

Due to the growing implementation of sensor-based battery management systems, nowadays it is possible to collect a considerable amount of time-stamped battery charging data in real time. This data is usually continuous measurements, including voltage, current, temperature and state of charge, measured during the charge, which are recorded during the charging cycle. This data is usually analyzed with machine learning methods and used to identify abnormal charging conduct. Nevertheless, most of the current methods are based on classical machine learning models only, and these might not be able to depict the complicated nonlinear relationships that can exist in multivariate time-series sensor data.

The new developments in quantum computing provided quantum-inspired machine learning (QML) methods that present alternative data representation and transformation techniques. Despite the fact that modern quantum devices have not offered any computational advantage of practical datasets, hybrid learning systems with classical and quantum-inspired components have been considered, because the feature representation can be improved. Here, it is assumed that QML will not replace the classical machine learning, but it represents another processing layer that can be added to the already existing learning pipeline.

This paper suggests a hybrid classical-quantum learning model to detect anomalies in EV battery charging systems with real time stamped sensor data. It is based on the suggested procedure that combines the classical time-series preprocessing and feature extraction with a quantum-inspired processing layer that is aimed at executing nonlinear feature transformation. The resulting hybrid feature representation is then studied with a classical classifier to determine abnormal charging patterns that relate to thermal runaway.

This project will have the main goal of designing and deploying an entire learning pipeline showing the extent to which advanced machine learning techniques can be integrated in a safety-critical and real world application. Instead of prioritizing the superior performance or computational acceleration, it focuses on the architectural design, feasibility, and the systematization of evaluation of hybrid feature representations of time-series anomaly detection. The efficiency of the suggested framework is assessed with the help of the data set of time-stamped sensor readings that were recorded in EV battery charge and thermal runaway events.
